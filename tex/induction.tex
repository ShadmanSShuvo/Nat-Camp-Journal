\chapter{Induction}% by Muhaiminul Islam Ninad}

\begin{linkb}
   \begin{itemize}
        \item \href{https://www.youtube.com/watch?v=C-0pRTD4H0w}{Induction (Ninad)} 
        \item \href{https://drive.google.com/file/d/1Z1h4jfZvrb69Jrayc0IMmopX9yzz9Z_L/view?usp=sharing}{First} \(+\) \href{https://drive.google.com/file/d/12zWftwXBXeYKMCBMGh-cPhG69r0zoc99/view}{Second}
   \end{itemize}
\end{linkb}

Today's Topic is Basic Induction. Our mentor started with showing an inductive proof of sum of positive integers:
\[ 1 + 2 +3+ \ldots + n = \frac{n(n+1)}{2} .\]
It is easy and left as an exercise.

\section{Induction/Weak Induction}
Our next problem is:

\begin{example}
Let, \(a_1=\sqrt{2}, a_{i+1}=\sqrt{2+a_i}\). Prove that, \[ a_n = 2\cos \frac{\pi}{2^{n+1}}. \]
\end{example}

\begin{soln}
Here we show that, base case is true as \(a_1=2\cos \frac{\pi}{2^2} = 2\cos \frac{\pi}{4} =2 \cdot \frac{1}{\sqrt 2} = \sqrt{2} \).

Now assume that it is true for \(a_k\) and \(a_k=2\cos \frac{\pi}{2^{k+1}}\). Then we will show that \(a_{k+1} = 2\cos \frac{\pi}{2^{k+2}}\).

\end{soln} 


\begin{example}
Show that for all positive integer \(n\), \[ 1+ \frac{1}{\sqrt{2}}+ \frac{1}{\sqrt{3}}+ \frac{1}{\sqrt{4}}+ \ldots \frac{1}{\sqrt{n}} \le 2\sqrt{n}. \]
\end{example}

\begin{soln}
Here, after showing the base case true, assume that it's true for integer $k$. Then for $k+1$, add $\frac{1}{\sqrt{k+1}}$ to both side.

After these, show that, $2\sqrt{k} + \frac{1}{k+1} \le 2\sqrt{k+1}$
Done!
\end{soln}

\begin{example}
Prove that for all positive integer \(n\), 
\[ \frac{1}{2} \cdot  \frac{3}{4} \cdot  \frac{5}{6} \cdot \cdots  \frac{2n-1}{2n} < \frac{1}{\sqrt{3n}} .\]
\end{example}

\begin{soln}
Here assume that it's tru for $k-1$ then show for $k$.

Then do some algebra which is left as exercise!
\end{soln}


\section{Strong Induction}
In this tecknique, we will assume that the statement is true for all integers from base case to \(k\) (in general).
Then we will show that its true for \(k+1\), and often we use the assertion that the statement is true for all integers from base case to \(k\).

\begin{example}
%Prove that any positive integer can be written as the sum of distinct and non-consecutive one or more Fibonacci Numbers.
The Fiobonacci sequence is defined as $F_1 = 1, F_2 = 2$ and for $n \ge 1$,
$F_{m+2} = F_{m+1} + F_m$. Show that every natural number n can be written as a sum of distinct
fiobonacci numbers, with no $2$ Fiobonacci numbers being consecutive ($F_n$ and $F_{n+1}$ are
consecutive).
\end{example}
%Note that Fibonacci numbers are \(F_1=1, F_2 = 1 , F_3 =3, \ldots F_n=F_{n-1} + F_{n-2} \).
\begin{soln}
We proceed by strong induction. Let $P(n)$ be the statement that n can be written
as a sum of distinct fiobonacci numbers with no $2$ consecutive. It is easy to verify $P(1)$
and $P(2)$. We now proceed by strong induction, so assume $P(i)$ for all $1 \le i \le n$. Now
consider the number $n + 1$ and the highest $F_m$ such that $F_m \le n + 1$. Then if $n + 1 = F_m$ we are done.
Otherwise, $n + 1 - F_m$ is a positive integer less than $n + 1$, and so $P(n + 1 - F_m)$ is
true by assumption. Therefore, we can write $n + 1 - F_m = F_{i_1} + F_{i_2} + \dots + F_{i_k}$ for distinct Fiobonacci numbers, no $2$ of which are consecutive. Now we have written $n + 1$ as a sum of distinct fiobonacci numbers
\[ n + 1 = F_{i_1} + F_{i_2} + \ldots + F_{i_k} + F_m.\]
such that no $2$ are consecutive execpt possibly the largest $2$, if $F_{i_k} = F_{m-1}$. But the latter case would imply that $n \le F_m + F_{m-1} = F_{m+1}$, contradicting the choice of $m$. This completes the induction.
\end{soln}
The above soln is from Jacob Tsimerman's \textit{A closer look at Induction} note.




\section{Real Induction}
\begin{example}
There are \(n\) lamps in a room, with certain lamps connected by wires. Initially
all lamps are off. You can press the on/off button on any lamp \(A\), but this also switches
the state of all the lamps connected to lamp \(A\) from on to off and vice versa. Prove that
by pressing enough buttons you can make all the lamps on. (Connections are such that if
lamp \(A\) is connected to lamp \(B\), then lamp \(B\) is also connected to lamp \(A\).)
\end{example}

\begin{soln}
We use induction on n. The base case of \(n = 1\) is trivial (just turn the lamp on).
Now assume the case of \(n - 1\) lamps. Now look at the set of n lamps and ignore some
lamp \(A\). Then by induction, we can turn the remaining lamps on by pressing the buttons
on some subset of them. Now if at the end of doing this \(A\) is also on, we are done. So we
can assume that at the end of doing this \(A\) is off. Since \(A\) was an arbitrary lamp, we can
assume that by pressing a sequence of buttons we can flip the states of all lamps except
one of our choosing. Now, taking \(A\) and \(B\) to be 2 different lamps and flipping the states
first of all lamps different from \(A\) and then all lamps different from B, we see that we can
flip the states of only \(A\) and \(B\). So this means we can flip the states of any number of even
lamps. Now we have 2 cases:
\begin{itemize}
	\item \(n\) is even: Then we are already done, since we can flip the states of any number of
even lamps.
	\item \(n\) is odd: In this case, there must be some lamp A connected to an even number of
lamps (prove this!) so first press the button on lamp \(A\). Now, including \(A\), an odd
number of lamps are on, so an even number of lamps that are off remain. Flip their
states to finish the proof!
\end{itemize}

\end{soln}
The above solution is from Jacob Tsimerman's \textit{A closer look at Induction} note.

Our class ended!

%\begin{problem}

%\end{problem}



\section{Practice Problems}


\begin{problem}

	\begin{hint}
		\addhint{ }
		\addhint{ }
	\end{hint}
\end{problem}

\begin{problem}

	\begin{hint}
		\addhint{ }
		\addhint{ }
	\end{hint}
\end{problem}

\begin{problem}

	\begin{hint}
		\addhint{ }
		\addhint{ }
	\end{hint}
\end{problem}

\begin{problem}

	\begin{hint}
		\addhint{ }
		\addhint{ }
	\end{hint}
\end{problem}

\begin{problem}

	\begin{hint}
		\addhint{ }
		\addhint{ }
	\end{hint}
\end{problem}

\begin{problem}

	\begin{hint}
		\addhint{ }
		\addhint{ }
	\end{hint}
\end{problem}

\begin{problem}

	\begin{hint}
		\addhint{ }
		\addhint{ }
	\end{hint}
\end{problem}


\begin{problem}

	\begin{hint}
		\addhint{ }
		\addhint{ }
	\end{hint}
\end{problem}
