\chapter{Harder Divisibility}% by Atonu Roy Chowdhury}
\begin{linkb}
   \begin{itemize}
        \item \href{https://www.youtube.com/watch?v=kf_0msAOTMc}{Harder Divisibility(Atonu)} 
        \item \href{https://drive.google.com/file/d/1ZURQO2KSR9aaeaVyalL7GNuVBYhOGswB/view?usp=sharing}{1} , 
			\href{https://drive.google.com/file/d/1QHrhEnmAzimZjt-8k_T8AJrX3KiRSm73/view?usp=sharing}{2} 
   \end{itemize}
\end{linkb}

%\textbf{Prerequisities:}
\subsection*{Prerequisities:}
\begin{enumerate}
	\ii Residue System
	\ii Reduced Residue System
	\ii Fermat's Little Theorem(FLT: $a^{p-1}\equiv 1 \pmod p$ where $(a,p)=1$) and Euler's Theorem($a^{\varphi (m)}\equiv 1 \pmod m$ where $(a,m)=1$ and $m$ is not necessarily prime). 
	\ii Wilson's Theorem($p$ prime $\Leftrightarrow (p-1)! \equiv -1 \pmod p$)
\end{enumerate}
All of these topics are covered in Basic Divisibility Class: \autoref{sec:divi}.

\section{Order}
By Fermat's Little Theorem and Euler's Totient Function, 
$$a^{\varphi (m)}\equiv 1 \pmod m$$
If we are going to calculate
$a^1, a^2, a^3, \ldots, a^{\varphi (m)} \pmod m$ we find that, $a^n \equiv 1 \pmod m$ where $m< \varphi(m)$.

As an example $2^3 \equiv 1 \pmod 7$ but $2^{\varphi (7)} \equiv 2^6 \equiv 1 \pmod 7$

Here we find that $3$ is more special than $6$ in our example. Actually $3$ is order of $2$ modulo $7$.
Which is denoted by $\ord_{7}(2)=3$.

\begin{definition}[Order]
$\ord_{m}(a)=\min \{x:a^x\equiv 1 \pmod m \}$
\end{definition}
here a point to be noted that $\gcd (a,m)=1$.
If they aren't co-prime then no power of $a$ wil be $1$ modulo $m$.
Which is left as an exercise for the readers.

\begin{theorem}[Fundamental Theorem of Orders]
$\ord_m(a)=d$ then, $d|N \Leftrightarrow a^N \equiv 1 \pmod m$.
\end{theorem}
Let assume $N=dk$ and we know $a^d\equiv 1 \pmod m$ then, we have $a^{dk} \equiv 1 \pmod m$.

Assume for the sake of contradiction $d\not\mid N \implies N=dq+r $ such that $0<r<d$.
 Now, $a^d\equiv 1 \pmod m$. as $d$ is the order.
 Then we have $a^{dq}\equiv 1 \pmod m \implies a^N = a^{dq+r}\equiv 1 \cdot a^r \pmod m$
 \[\implies a^r \equiv 1  \pmod m \]
 which is a contradiction as $r$ can't be the order as $r<d$.

Here, we learn:
\begin{enumerate}
	\ii $d\mid \varphi_(m)$ which means $d\in \{ \mathrm{divisors of \varphi_(m)}\}$
	\ii $a^x\equiv a^y \pmod m \Leftrightarrow x\equiv y \pmod d$ where $\gcd(m,n) =1$.
	WLOG, $x\ge y$ so, $a^{x-y} \equiv 1 \pmod m$ then, $d\mid x-y $ and then we get $x\equiv y \pmod d$.

\end{enumerate}

\begin{example}
Given that $\gcd (a,m)=1$. Prove that $n\mid \varphi (a^n-1)$.
\end{example}
Let, $N=a^n-1$, so, $n\mid \varphi (N)$ and $\ord_N(a)\mid \varphi(N)$

So, we have to show that $a^n\equiv 1 \pmod N$

Assume for the sake of contradiction that, $k<n$ then we have, $a^k\equiv 1 \pmod N \implies N\mid a^k-1$

But, which means, $a^k-1 \ge N= a^n-1$

Contradiction as we assumed that $k<n$.

\begin{example}
Find all $n$ $n\mid 2^n-1$.
\end{example}
Here we find that $n$ must be odd and $n=1$ works. 
We let $n=2k+1$, so, $2k+1 \mid 2^{2k+1} -1$ 

Here we will use a lemma.
\begin{lemma}
If $a^x \equiv 1, a^y \equiv 1 $ all are taken modulo $m$. Then we have $a^{\gcd (x,y)} \equiv 1 \pmod m$
\end{lemma}

Let, $p$ be the smallest prime factor of $n$ (this trick is called the smallest prime factor(spf trick) of $n$)

$d=\ord_p(2)\implies d\mid \varphi(p)=p-1$
By FLT, $p\mid n \mid 2^n-1 \implies 2^n \equiv 1 \pmod p \implies d\mid n$ .

But then, 
\[ d\mid p-1, \ \ \ d\mid n \]
which is a contradiction! 
\section{Primitive Root}
\begin{definition}
Let $g$ is a positive integer and $\gcd(g,n)=1$. Then we shall call $g$ "Primitive Root modulo $m$" if $\ord_m(g)=\varphi(m)$.
\end{definition}
As an example, $\gcd(3,7)=1$, $3^6 \equiv 1 \pmod 7$.
We also know that $\varphi(7)=6$.
So, we can say that $3$ is a primitive root modulo $7$.

\begin{theorem}[Primitive Root]
Let, $p$ be a prime. Then $\exists g \in \{ 1,  \ldots, p-1 \}$ such that $g$ is a primitive root modulo $p$.
\end{theorem}
\begin{remark}
$\{1,\ldots, p-1\}$ is the reduced residue system of $p$ denoted by $RRS(p)$.
\end{remark}

\textbf{Usage and Properties of Primitive Root:}

%\begin{enumerate}
%	\ii
\begin{prop}
If $g$ is a primitive root modulo $m$ and $\varphi(m)$ is even then,$g^{\varphi(m)/2}\equiv -1 \pmod m$.
\end{prop}

Let, $\varphi(m)=2n$ So we have, \[g^{2n}\equiv 1 \pmod m \implies m\mid g^{2n}-1=(g^n+1)(g^n-1)\]
Which means, $m\mid g^n+1 $ or $m\mid g^n-1$.
So we have, $g^n\equiv -1$ or $g^n\equiv +1$ modulo $m$.

Now, AFTSOC (Assume for the sake of contradiction) that 
$g^n \equiv 1 \pmod m$
Then, $\varphi(m)=\ord_m(g)\le n$

But now we get $2n\le n$ which is absurd.

\begin{prop}
$S=\{ g, g^2, g^3, \ldots, g^{\varphi(m)} \}$ is $RRS(m)$ where $g$ is primitive root modulo $m$.
\end{prop}

%proof goes here



\section{Wilson's Theorem}

\begin{theorem}[Wilson's Theorem]
If $p$ is a prime then $(p-1)!\equiv -1 \pmod p$.
\end{theorem}
You can find a proof from Brilliant.org

\begin{problem}
Let $p$ be an odd prime. Find all $k$ such that \[p\mid 1^k + 2^k + 3^k +\ldots + (p-1)^k\]
\end{problem}
Here solving the above problem, you assume that you don't know any formula.