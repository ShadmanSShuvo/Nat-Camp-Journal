


%%%%%%%%%%%%%%%%%%%%%%%
\chapter{Combinatorics TSTST}\label{tstst:combi}
%\chapter{April 30, 2021}
%\begin{center}
%\LARGE{THIS IS OUR OFF DAY} 

%\LARGE{:D}

%\end{center}

%Problems---
%Exam-1 Problems
%\includepdf[pages=-]{exam1.pdf}
The Combinatorics Exam of The National Camp.
Problems are given below:

\begin{prob}\label{combi:1}
You have a set $S$ of $19$ points in the plane such that given any three points in $S$, there exist two of them
whose distance from each other is less than $1$. Prove that there exists a circle of radius $1$ that encloses
at least $10$ points of $S$.
	\begin{hint}
	\addhint{Start with Pigeon-hole Principle!}
	\addhint{}
	\end{hint}
\end{prob}

\begin{prob}\label{combi:2}
There are $2021$ stones in a pile. At each step, Lazim chooses a pile with at least two stones, splits it into
two piles, and multiplies the sizes of the resulting two piles. He keeps doing this until there are $2021$
piles each containing exactly one stone. Finally, he adds up all the products he has obtained during the
process and ends up with the number $N$ . Find, with proof, all the possible values of $N$ .
	\begin{hint}
	\addhint{}
	\addhint{}
	\end{hint}
\end{prob}

\begin{prob}\label{combi:3}
Show that for all positive integers $n\ge 2$, you can cut any quadrilateral into $n$ isosceles triangles.
	\begin{hint}
	\addhint{Induction!}
	\addhint{}
	\end{hint}
\end{prob}

\begin{prob}\label{combi:4}
Let $n \ge 1$ be an integer. A non-empty set is called “good” if the arithmetic mean of its elements is an
integer. Let $T_n$ be the number of good subsets of $\{1, 2, 3 \ldots , n\}$. Prove that for all integers $n$, $T_n$ and $n$
leave the same remainder when divided by $2$.
	\begin{hint}
	\addhint{}
	\addhint{}
	\end{hint}
\end{prob}

\begin{prob}\label{combi:5}
We place some checkers on an $n \times n$ checkerboard so that they follow the conditions:
	\begin{itemize}
		\ii every square that does not contain a checker shares a side with one that does;
		\ii given any pair of squares that contain checkers, we can find a sequence of squares occupied by
		checkers that start and end with the given squares, such that every two consecutive squares of the
		sequence share a side.
	\end{itemize}
Prove that at least $(n^2-2)/3$ checkers have been placed on the board.
	\begin{hint}
	\addhint{Extremal Combinatorics!}
	\addhint{Call a square good if it contains a checker or shares a side with a square containing a checker.}
	\addhint{We don't know how much contribution a checker has. }
	%\addhint{So, we should not }
	\addhint{So, we start adding checkers. Suppose that $k$ checkers will be added.}
	\addhint{Place one checker on the board to start, and then in each step place one checker adjacent to one that has already been placed.}
	\addhint{In first step, after adding $2$ checkers, at most $5$ squares become good.}
	\addhint{In each step, at most $3$ square which were not already good, become good.}
	\addhint{So, $5+3(k-2) = 3k+2$ and hence, $n^2 \le 3k+2$.}
	\end{hint}
\end{prob}


Full Solutions to these problems can be found at \autoref{sols:tstst}.