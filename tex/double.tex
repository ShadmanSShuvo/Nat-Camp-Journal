\chapter{Double Counting and PIE}% by Nishat Anjum Bristy}

\begin{linkb}
   \begin{itemize}
        \item \href{https://www.youtube.com/watch?v=-XuswFtIiRY}{1} + \href{https://www.youtube.com/watch?v=UkvhVQ1_QEo}{2}
        \item \href{https://drive.google.com/file/d/1D-E7FSLqfIj8gYf0Dmg94uEr8pROjN-e/view}{1} , \href{https://drive.google.com/file/d/16tu7PZ_95BnBYSLIJarznwacp3deTxRF/view}{2}
        \item \href{https://drive.google.com/file/d/1FQlCRhMpyAYWvRIxWtAJGNN7kuQA3k_6/view}{P set}
   \end{itemize}
\end{linkb}

\section{Double Counting}
Double counting is also known as counting in two ways. By this technique, we count something in two ways!

Here is our first example:
\begin{example}
Show that, \[1+2+3+\ldots +n=\frac{n(n+1)}{2}\]
\end{example}
You can easily solve this by induction. but here we present a solution with double counting.

By writing the left side as $S_n$ we proceed as follows:
\[S_n = 1+2+\ldots +n\]
\[S_n = n+(n-1)+\ldots+1\]
Thus, \[S_n=\frac{n(n+1)}{2}.\]

But now, we want to think the right side of 
the equation in combinatorial arguments. 
Here we see that the right side is equal to 
the number of ways to choose $2$ things from 
$n+1$ given things. If we can establish a 
bijection between them then we are done!

Here we take two empty cells, and we want to place the numbers in the cells. 
Suppose the cells:
\[\boxed{i} \boxed{j}\]
such that, $i<j$. 
Now if we place $i=1$ then we see that, there are $n$ choice for $j$. By similar arguments, we calculate, 
\[\binom{n+1}{2}=1+2+3+\ldots+n.\]



\begin{example}
In a given series th esum of any $7$ consecutive number is negative and the sum of any $11$ consecutive numbers is positive.
Prove that the series can't have $17$ terms.
\end{example}


Here we make a matrix of the numbers of the series by writing consecutive terms.
%\begin{center}
%\begin{align*}
%t_1 &t_2 &t_3 &\ldots &t_{11}\\
%t_2 &t_3 &t_4 &\ldots &t_{12}\\
%\ldots &\ldots &\ldots &\ldots &\ldots \\
%\ldots &\ldots &\ldots &\ldots &\ldots \\
%t_7 &t_8 &t_9 &\ldots &t_{17}
%\end{align*} 
%\end{center}
%\begin{tabular}
%$t_1$ &$t_2$ &$t_3$ &$\ldots$ &$t_{11}$\\
%$t_2$ &$t_3$ &$t_4$ &$\ldots$ &$t_{12}$\\
%$\ldots$ &$\ldots$ &$\ldots$ &$\ldots$ &$\ldots$ \\
%$\ldots$ &$\ldots$ &$\ldots$ &$\ldots$ &$\ldots$ \\
%$t_7$ &$t_8$ &$t_9$ &$\ldots$ &$t_{17}$
%\end{tabular} 

%\begin{center}
%\begin{array}{cc}
%\hline
%t_1 & t_2 &t_3 &\ldots &t_{11}\\
%\hline
%t_2 & t_3 &t_4 &\ldots &t_{12}\\
%\hline
%\ldots &\ldots &\ldots &\ldots &\ldots \\
%\hline
%\ldots &\ldots &\ldots &\ldots &\ldots \\
%\hline
%t_7 &t_8 &t_9 &\ldots &t_{17}\\
%\hline
%\end{array} 
%\end{center}

%\begin{array}
%$t_1$ &$t_2$ &$t_3$ &$\ldots$ &$t_{11}$\\
%\hline
%$t_2$ &$t_3$ &$t_4$ &$\ldots$ &$t_{12}$\\
%\hline
%$\ldots$ &$\ldots$ &$\ldots$ &$\ldots$ &$\ldots$ \\
%\hline
%$\ldots$ &$\ldots$ &$\ldots$ &$\ldots$ &$\ldots$ \\
%\hline
%$t_7$ &$t_8$ &$t_9$ &$\ldots$ &$t_{17}$
%\end{array} 
\begin{center}
\[
\begin{matrix}
t_1 & t_2 & t_3 & \ldots & t_{11}\\
t_2 & t_3 & t_4 & \ldots & t_{12}\\
\ldots & \ldots & \ldots & \ldots & \ldots \\
\ldots & \ldots & \ldots & \ldots & \ldots \\
t_7 & t_8 & t_9 & \ldots & t_{17}
\end{matrix} 
\]
\end{center}
Then sum up all the rorws and all the columns. See that, the rows sum up to a positive number but the columns sum up to a negative number.
Which is a contardiction!
\begin{remark*}
In the above example, we used the Fubini's Principle.
\end{remark*}
\begin{example}
Determine the total number of fixed points in all the permutations of \[ \{1, 2, 3, \ldots , n\}\]
\end{example}
Here observing with smaller $n$ you can find the answer. It is just $n!$.

Try to make a table here. You probably find that, in each column there is $(n-1)!$ number of fixed points.
So the total is $n(n-1)!=n!$.



%\begin{example}
%\ldots
%\end{example}


%\begin{example}
%There is a set $S=\{1,2,3,\ldots, mn\}$ and another set $T$ such that, $|T|=2n$ and each element of $T$ is a $m$ element subset of $S$.
%In every two elemnts of $T$ there is at most one common element.
%And, every element of $S$ lies exactly in two %element
% of $T$. 
% Prove that, $m\le 2n-1$.
%\end{example}
%\section{}
%\ldots
\section{Principle of Inclusion-Exclusion}
$A,B$ are two set and we know that $|a\cup B|=|A|+|B|-|A\cap B|$. This called inclusion exclusion pirinciple.

For three sets $A,B,C$ we have \[|A\cup B \cup C| = |A| + |B| + |C| -|A\cap B| - |B\cap C| - |C\cap  A| + |A\cap B\cap C|\]

For larger number of sets we have: 
\[| A_1 \cup A_2  \cup \ldots A_n |=|A_1| + |A_2| + \ldots +|A_n| -|A_1 \cap A_2| - |A_2 \cap A_3| - \ldots  -|A_{n-1} \cap A_n| + |A_1 \cap A_2 \cap A_3| + \ldots + \ldots  \]
So positive for evens and negative for odds.

The above identity can be written as 
\[|A_1 \cup A_2 \cup \ldots \cup A_n = \sum_{k=1}^n \sum_{a_i \neq a_j} (-1)^{k+1} |S_{a_1}\cap \ldots \cap S_{a_k} \]

In problem solving it is also an useful trick!