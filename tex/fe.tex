\chapter{Functional Equation}% by Raiyan Jamil}

\begin{linkb}
   \begin{itemize}
        \item \href{https://www.youtube.com/watch?v=7QhswMDYy0Y}{Functional Equation (Raiyan)}
        \item \href{https://drive.google.com/file/d/1KknJbbhPo3nK4ZneB3e22_bxlnGXF5j0/view}{Class} , \href{https://drive.google.com/file/d/1Kr_K2i_SQcefIWVR1yvgNN0Pnd3ZG0sg/view}{Chen}
        \item \href{https://drive.google.com/file/d/1cf1tifSEqKebrrDLyk2DONn0-0VlBSWs/view}{P set}
   \end{itemize}
\end{linkb}

(You can follow the note: \textit{Introduction to Functional Equation} by Evan Chen)



\section{Definitions}
Let $f : A \to B$ be a function.
The set $A$ is called the \vocab{domain}, and $B$ the \vocab{codomain}.
A couple definitions which will be useful:

\begin{definition}
	A function $f : A \to B$ is \vocab{injective} if $f(x) = f(y) \iff x=y$.
	(Sometimes also called \emph{one-to-one}.)
\end{definition}
\begin{definition}
	A function $ : A \to B$ is \vocab{surjective} if for all $b \in B$,
	there is some $x \in A$ such that $f(x) = b$.
	(Sometimes also called \emph{onto}.)
\end{definition}
\begin{definition}
	A function is \vocab{bijective} if it is both injective and surjective.
\end{definition}

\section{1st example}
\begin{example}
Find all functions $ f:\mathbb{R}\to\mathbb{R} $ such that $ f(f(x)^2+f(y)) = xf(x)+y $,$ \forall x,y\in R $.
\end{example}

Taking $x=y=0$, we see $f(a)=0$ for $a=f(0)^2+f(0)$. Taking $x=a$ gives $f(f(y))=y$, which means $f$ is bijective. Since $f$ is bijective, $f(a)=0=f(f(0))\implies f(0)=0$. Taking $y=0$ and $x\mapsto f(x)$ yields $f(x^2)=f(x)f(f(x))=xf(x)$. Swapping the sign on $x$, we see $f$ is odd. Making the replacements, $x\mapsto f(x)$, and $y\mapsto f(y)$, gives$$f(x^2+y)=xf(x)+f(y)=f(x^2)+f(y).$$Now let $x\ge 0$, and note that
\begin{align*}
f(x^2+2x+1)&=(x+1)f(x+1)\\
f(x^2)+2f(x)+f(1)&=(x+1)(f(x)+f(1))\\
f(x)&=xf(1)
\end{align*}But, $f$ odd extends this to all reals. Noting $f$ is involutive, we see that $f(1)=\pm 1$. This gives the solutions $\boxed{f=x}$ and $\boxed{f=-x}$

\section{Cauchy's Functional Equation}
\begin{theorem}[Cauchy's Functional equation over $\mathbb{Q}$]
If $ f(a)+f(b)=f(a+b) $ over $\mathbb{Q}$, then $f(x)=kx$ are only solutions for it. 
\end{theorem}
Let $f(1)=k$

Setting $x=y=0$, we get $f(0)=0$. and $x=y=1$, we get $2f(1)=f(2)=2k$.
Now setting $(x,y)=(x,1)$, we get $f(x)+f(1)=f(x+1) \implies f(x+1)=kx+k=k(x+1)$, so we are done for integers.

We can show that $q\times f(\frac{p}{q})=(p)=kp$.

from where we get that $f(p/q)=\frac{kp}{q}$ and we are done over $\mathbb{Q}$

\begin{example}[IMO 2019 P1]
Let $\mathbb{Z}$ be the set of integers. Determine all functions $f: \mathbb{Z} \rightarrow \mathbb{Z}$ such that, for all integers $a$ and $b$,$$f(2a)+2f(b)=f(f(a+b)).$$
\end{example}
Set $a = 0$ to get $f(f(b)) = 2f(b) + f(0)$. Thus the given rewrites as $f(2a) + 2f(b) = 2f(a + b) + f(0)$. Now let $a = b$ to get $f(2a) = 2f(a) - f(0)$, so the given becomes $f(a) + f(b) = f(a + b) + f(0)$. Letting $g(x) = f(x) - f(0)$ this is just $g(a + b) = g(a) + g(b) $ over integers, implying cauchy, we get $g(x) = kx$ and $f(x) = kx + f(0)$

Now doing some substitutions, we get $k=2$
So, either $f(x)=0$ for all $x$ or $f(x)=kx+c$ 
\begin{theorem}[Cauchy + Continuous $\implies$ Linear]
	Suppose $f : \RR \to \RR$ satisfies $f(x+y) = f(x) + f(y)$.
	Then $f(qx) = qf(x)$ for any $q \in \QQ$.
	Moreover, $f$ is linear if any of the following are true:
	\begin{itemize}
		\ii $f$ is continuous in any interval. 
		\ii $f$ is bounded (either above or below) in any nontrivial interval.
		\ii There exists $(a,b)$ and $\eps > 0$ such that $(x-a)^2 + (f(x)-b)^2 > \eps$ for every $x$
		(i.e.\ the graph of $f$ omits some disk, however small).
	\end{itemize}
\end{theorem}

\begin{example}
Find all continuous function $f: \RR \to \RR$ such that $$f(x+y)-f(x)-f(y)=xy(x+y)$$
\end{example}

\begin{example}[Field Automorphisms of $\RR$]
	Solve over $\RR$:
	\[ f(x+y) = f(x) + f(y) \quad\text{and}\quad f(xy) = f(x)f(y). \]
\end{example}
\begin{proof}
	[Solution]
	We claim $f(x) = x$ and $f(x) = 0$ are the only solutions (which both work).
	According to the theorem, to prove $f$ is linear it suffices to show $f$ is nonnegative
	over some nontrivial interval.
	Now, \[ f(t^2) = f(t)^2 \ge 0 \] for any $t$,
	meaning $f$ is bounded below on $[0,\infty)$
	and so we conclude $f(x) = cx$ for some $c$.
	Then $cxy = (cx)(cy)$ implies $c \in \{0,1\}$, as claimed.
\end{proof}

\section{Some intuitions and motivations}
\begin{example}[Cancelling the stuffs]

	Solve over $\RR$: \[ f(x^2+y) = f(x^{27} + 2y) + f(x^4). \]
\end{example}
\begin{soln}
	For this problem, we claim the only answer is the constant function $f=0$.
	As usual our first move is to take the all-zero setting, which gives $f(0) = 0$.

	Now, let's step back: can we do anything that will make lots of terms go away?
	There's actually a very artificial choice that will do wonders.
	It is motivated by the \vocab{battle cry}:
	\begin{quote}
		\itshape ``DURR WE WANT STUFF TO CANCEL.''
	\end{quote}
	So we do the most blithely stupid thing possible.
	\emph{See that $x^2+y$ and $x^{27}+2y$ up there?}
	Let's make them equal in the rudest way possible:
	\[ x^2 + y = x^{27} + 2y \iff y = x^2 - x^{27}. \]
	Plugging in this choice of $y$,
	this gives us $f(x^4) = 0$, so $f$ is zero on all nonnegatives.

	All that remains is to get $f$ zero on all reals.
	The easiest way to do this is put $y=0$ since this won't 
	hurt the already positive $x^2$ and $x^4$ terms there.
\end{soln}
This is a common trick: see if you can make a substitution that will kill off two terms.

\begin{example}[Finding a constant term]
	\label{ex:singapore}
	Solve over $\RR$:
	\[ (x-y)f(x+y) - (x+y)f(x-y) = 4xy(x^2-y^2). \]
\end{example}
\begin{soln}
	First of all, the $x-y$ and $x+y$ everywhere are a mess,
	so we replace them with $a = x+y$ and $b = x-y$
	(this doesn't lose any information).
	Then $x^2-y^2 = ab$, and also $2x = a+b$, $2y = a-b$.
	So, the equation is just saying that
	\[ bf(a) - af(b) = ab(a^2-b^2). \]

	Plugging  $a=0$ is fruitful: we derive
	$bf(0) = 0$, so $f(0) = 0$.
	
	Next, assume $a,b \neq 0$.
	We do the following nice trick:
	if we divide by $ab$, we obtain
	\[ \frac{f(a)}{a} - \frac{f(b)}{b} = a^2 - b^2. \]
	You'll notice that the $a$ and $b$ terms are completely isolated now!
	That is, we can write
	\[ \frac{f(a)}{a} - a^2 = \frac{f(b)}{b} - b^2. \]
	This is enough to imply that $\frac{f(a)}{a} - a^2 = c$ for some constant $c$,
	whenever $a \neq 0$; hence $f(x) = x^3 + cx$ for some $c$.
	Initially we only know this for $x \neq 0$,
	but it holds for $x=0$ as well since $f(0) = 0$.

	And now, we observe that these solutions all work, and we're done.
\end{soln}

\section{Three More Tricks}
Here are three more tricks that are frequently useful.

\begin{itemize}
	\ii \vocab{Tripling an involution}.
	If you know something about $f(f(x))$,
	try applying it $f(f(f(x)))$ in different ways.
	For example, if we know that $f(f(x)) = x+2$,
	then we obtain $f^3(x) = f(x+2) = f(x)+2$.

	\ii \vocab{Isolated parts}.
	When trying to obtain injective or surjective,
	watch for ``isolated'' variables or parts of the equation.
	For example, suppose you have a condition like
	\[ f(x+2xf(y)^2) = yf(x) + f(f(y)+1) \]
	
	Noting that $f \equiv 0$ works, assume $f$ is not zero everywhere.
	Then by taking $x_0$ with $f(x_0) \neq 0$,
	one obtains $f$ is injective.
	(Try putting in $y_1$ and $y_2$.)

	Proving surjectivity can often be done in similar spirit.
	For example, suppose
	\[ f(f(y)+xf(x)) = y+f(x)^2. \]
	By varying $y$ with $x$ fixed we get that $f$ is surjective,
	and thus we can pick $x_0$ so that $f(x_0) = 0$
	and go from there.
	Surjectivity can be especially nice if every $y$
	is wrapped in an $f$, say; then each $f(y)$ just becomes
	replace by an arbitrary real.

	\ii \vocab{Exploiting ``bumps'' in symmetry}.
	If some parts of an equation are symmetric and others are not,
	swapping $x$ and $y$ can often be helpful.
	For example, suppose you have a condition like
	\[ f(x+f(y)) + f(xy) = f(x+1)f(y+1) - 1 \]

	This equation is ``almost symmetric'',
	except for a ``bump'' on the far left where $f(x+f(y))$ is asymmetric.
	So if we take the equation with $x$ and $y$ flipped
	and then eliminate the common terms,
	we manage to obtain \[ f(x+f(y)) = f(y+f(x)). \]
	If we've shown $f$ is injective, we are even done!
	So often these ``bumps'' are what let you solve a problem.
	(In particular, don't get rid of the bumps!)
\end{itemize}


\section{Practise Problems}
\begin{problem}
Find all continuous functions $f: \mathbb{R} \to \mathbb{R}$ such that $$f(f(x+y))=f(x)+f(y)$$
\begin{hint}
\addhint{plugging in $(x+y,0)$ helps}
\addhint{Find $f(0)$}
\addhint{use cauchy}
\end{hint}
\end{problem}

\begin{problem}
Find all functions $f: \mathbb{R} \to \mathbb{R}$ such that $$f(f(x+y))=f(x+y)+f(x)f(y)-xy$$ for all $x,y \in \mathbb{R}$
\begin{hint}
\addhint{}
\end{hint}
\end{problem}

\begin{problem}
Find all functions $f: \mathbb{R} \to \mathbb{R}$ such that $$f(x+2y)=f(x)(2y+1)+f(xy)$$ for all $x,y \in \mathbb{R}$
\begin{hint}
\addhint{}
\end{hint}
\end{problem}

\begin{problem}
Find all continuous functions $f: \mathbb{R} \to \mathbb{R}$ such that $$f(x+af(y))=f(x+y)+(a-1)y$$ for $a\neq 0,1$ is a constant.
\begin{hint}
\addhint{}
\end{hint}
\end{problem}
\begin{problem}
Find all functions $f: \mathbb{R} \to \mathbb{R}$ such that $$f(x^2+y)=f(f(x)-y)+4f(x)y$$ for all $x,y \in \mathbb{R}$
\begin{hint}
\addhint{}
\end{hint}
\end{problem}