\chapter{Quadratic Residue and Diophantine Equation}% by Atonu Roy Chowdhry}
\begin{linkb}
   \begin{itemize}
        \item \href{https://www.youtube.com/watch?v=YeurXbh2sIQ}{Video 1} + \href{https://www.youtube.com/watch?v=3ZBGv6dH37o}{Video 2}
        \item \href{https://drive.google.com/file/d/1lHBgt42uKI6qc0-Yt_Sx7boahxsYLVN7/view}{An Intro ...(book)}, \href{https://drive.google.com/file/d/12nate-JVlXwJUkMS7ja6ftBvyTWatace/view?usp=sharing}{Class note}
        
        \item \href{https://drive.google.com/file/d/16tC_XxRU31j_iOVZ3gEb_BMcZlQBfIP7/view?usp=sharing}{Note + P Set}
   \end{itemize}
\end{linkb}
\section{Quadratic Residue}

(Note: here \textit{qr} means quadratic residue)

\begin{definition}[Quadratic Residue]
$a$ is called quadratic residue modulo $n$ if $\exists x: x\equiv a \pmod n$. 
\end{definition}

\begin{definition}[Quadratic Residue Class]
$qr(n)=\{a:a \mathrm{is \ a \ qr \ in \ mod} n \}$.
\end{definition}
As an example $qr(5)=\{0,1,4\}$.

\begin{theorem}
If $-1$ is a qr in mod $p$ where $p$ is an odd prime. Then, $ p \equiv 1 \pmod 4$.
\end{theorem}

AFTSOC, $p\equiv -1 \pmod 4$ or $p = 4k-1$.
So $\exists x : x^2\equiv -1 \pmod p$. 
\[x^{p-1}\equiv 1 \pmod p \ \mathrm{By FLT}\]
\[\implies x^{4k-2}\equiv 1\pmod p\]
\[\implies (x^2)^{2k-1}\equiv 1 \pmod p\]
\[\implies (-1)^{2k-1}\equiv 1 \pmod p\]
\[-1\equiv 1 \pmod p\]
which is absurd.

The converse of the above theorem is also true.
\begin{theorem}
If $ p \equiv 1 \pmod 4$ is a prime then -1 is a quadratic residue mod $p$
\end{theorem}
The proof is constructive\ldots


\section{Legendre's Formula}
\begin{definition}[Legendre Symbol]
\[\left(\frac{n}{p}\right)=\begin{cases}
				0 & p\mid n \\
				1 & n\in qr(p) \\
				-1 & n\not\in qr(p)
				\end{cases}
				\]
\end{definition}

\begin{theorem}
\[a^{\frac{p-1}{2}}\equiv \left(\frac{a}{p}\right) \pmod p \]
\end{theorem}


\begin{lemma}
\[\left(\frac{ab}{p}\right)=\left(\frac{a}{p}\right)\left(\frac{b}{p}\right)\]
\end{lemma}
The meaning of the above lemma is if $a,b$ are both qr then $ab$ is also qr.

If $a,b$ both are not qr then $ab$ is a qr.

But, if only one of them is qr, then, $ab$ is not qr.

\begin{lemma}
\[\left(\frac{-1}{p}\right)=\begin{cases}
			-1 & p=4k+1 \\
			1 & p=4k-1
			\end{cases}
			\]
\end{lemma}

\begin{lemma}
\[\left(\frac{2}{p}\right)=(-1)^{\frac{p-1}{8}}\]
\end{lemma}


\begin{lemma}
\[\left(\frac{p}{q}\right)\left(\frac{q}{p}\right)=(-1)^{\frac{p-1}{2}}\cdot (-1)^{\frac{q-1}{2}}\]
\end{lemma}
\textbf{Techniques:}
Whenever you see squares you should take moulo $2^n$.

\section{Diophantine Equation}
Here we are going to solve some diophantine equation problems (i.e. problems asking for integer solutions)
\begin{example}
$d\neq 2,5,13$, $S=\{2,5,13,d\}$. Prove that, $\exists a,b \in S$ such that $ab-1 \neq k^2$
\end{example}
Here you have to show that there is no such $d$ that satisfies all the equations:
\begin{align*}
2d-1 &=x^2 \\
5d-1 &=y^2 \\
13d-1 &=z^2 
\end{align*}

AFTSOC, exists such $d$.  
First take mod $4$,

$qr(4)=\{0,1\}$

So, $2d-1 \pmod 4 \in \{0,1\}.$ Here, $2d-1$ doesn't congruent to 0 modulo $4$ as it is odd.

So, $2d-1 \equiv 1 \pmod 4 \implies d\equiv 1 \pmod 2$

But, then mod $4$ and mod $8$ aren't useful. Then take mod 16. Then some exercise left for the readers.


%\paragraph{What if power $\ge 2$?}
\begin{example}
Find all solutions to the equation:
\[x^3 + y^4 =7\]
\end{example}
Here we need to find a prime $p$ such that $3|p-1$ and $4|p-1$. 
$p=13 $ is a good choice.

So we work with mod $13$. We get:
$x^3 \pmod {13} =\{0,1,5,8,12\}$ and $y^4 \pmod {13} = \{0,1,3,9\}$.

But no such summation of two elements from the sets is equal to $7$. So, there is n such solution.

\begin{example}
Find all solutions to the equation:
$$x^5-y^2=4$$
\end{example}
Here the magical mod is $11$.
We get $x^5 \pmod {11} =\{0,1,-1\}$. And $y^2=x^5-4 \pmod {11} =\{6,7,9\}$

But quadratic residue $\pmod {11} =\{0,1,4,9,5,3\}$.

So, there is no such solution.

\section{Practise Problems}

\begin{problem}


	\begin{hint}
	\addhint{}
	\addhint{}
	\end{hint}
\end{problem}

\begin{problem}


	\begin{hint}
	\addhint{}
	\addhint{}
	\end{hint}
\end{problem}

\begin{problem}


	\begin{hint}
	\addhint{}
	\addhint{}
	\end{hint}
\end{problem}

\begin{problem}


	\begin{hint}
	\addhint{}
	\addhint{}
	\end{hint}
\end{problem}

\begin{problem}


	\begin{hint}
	\addhint{}
	\addhint{}
	\end{hint}
\end{problem}

\begin{problem}


	\begin{hint}
	\addhint{}
	\addhint{}
	\end{hint}
\end{problem}

\begin{problem}


	\begin{hint}
	\addhint{}
	\addhint{}
	\end{hint}
\end{problem}