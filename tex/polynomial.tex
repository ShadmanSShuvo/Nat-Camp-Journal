\chapter{Polynomials}% by Mursalin Habib}

\begin{linkb}
   \begin{itemize}
        \item \href{https://www.youtube.com/watch?v=w52vpcuTAwo}{Polynomial (Mursalin)}
        \item \href{https://drive.google.com/file/d/1SoVQm_j-osYbXPg7ZHeS3hQdGCw9m0Yo/view}{Polynomials Pset}
   \end{itemize}
\end{linkb}


What is a polynomial? Well, an example of a polynomial is given below:
\[x^2-5x+6.\]
Generally a polynomial is a form of:
\[a_nx^n + a_{n-1}x^{n-1} + a_{n-2}x^{n-2}+ \ldots + a_2x^2 + a_1x + a_0 .\]
where $a_n\neq 0$. Also assume that $a_n, a_{n-1}, \ldots$ are generally real numbers. 
%otherwise  
Here comes some definitions:
\begin{definition}[Coefficient]
$a_n,a_{n-1}, \ldots$ are called coefficients.
\end{definition}
\begin{definition}[Leading Coefficient]
$a_n$ is called the leading coefficient.
\end{definition}
\begin{definition}[Degree of a polynomial]
Maximum power $n$ is called the degree of a polynomial.
\end{definition}
\begin{definition}[Monic Polynomial]
A polynomial with leading coefficient $1$ is called a monic polynomial.
\end{definition}
\begin{definition}[Root of polynomial]
For a polynomial, $P(x)=a_nx^n + a_{n-1}x^{n-1} + a_{n-2}x^{n-2}+ \ldots + a_2x^2 + a_1x + a_0$. Roots of polynomial are the set of variables for which $P(x)=0$.
\end{definition}
\begin{theorem}[Fundamental theorem of Algebra]
For any n-degree polynomial, there are at most n distinct roots.
\end{theorem}
Here we are introduced with some techniques to solve polynomial problems:
\section{Exploiting Symmetry}
\begin{example}
Find all real $x$ that satisfy the equation
\[(x-1)(x^2+1)(x^3+1)=30x^3.\]
\end{example}
Here do some algebra to get:
\[x^6 + x^5 + x^4 -28x^3 +x^2 +x+1=0.\]
It is a polynomial of degree $6$ which is messy.
But you can see that here is a symmetry left to the term $-28x^3$ with the right part of the equation. 
Also observe that, if $x$ is a root then $\frac{1}{x}$ is also a root. 
So, we are going to divide the equation by $x^3$ and we get:
\[x^3+x^2 +x -28 +\frac{1}{x}+\frac{1}{x^2}+\frac{1}{x^3}=1.\]
Here we pair up $x^3$ with $\frac{1}{x^3}$ and so on. We get the following: 
\[\left(x^3+\frac{1}{x^3}\right)+\left(x^2+\frac{1}{x^2}\right)+\left(x+\frac{1}{x}\right)-28=0.\]
Now we assume $x+\frac{1}{x}=y$ and you see that, $x^3+\frac{1}{x^3}=y^3-3y, x^2+\frac{1}{x^2} =y^2-2y, x+\frac{1}{x}=y$ and we proceed by following:
\begin{align*}
\left(x^3+\frac{1}{x^3}\right)+\left(x^2+\frac{1}{x^2}\right)+\left(x+\frac{1}{x}\right)-28&=0\\
y^3 -3y +y^2 -2y +y-28 &=0\\
y^3 +y^2 -2y-30 &=0
\end{align*}
But it is a polynomial with degree $3$, which is also not so easy to solve. But we have a theorem for this: Rational Root Theorem.
%\begin{theorem}[Rational Root Theorem]
%A polynomial $P(x)$ has a rational root $\frac{p}{q}$ if and only if $p|a_0$ and $q|a_n$. 
%\end{theorem} 
if we can find a root of our polynomial then it comes easy to solve a quadratic eqaution.

After playing with numbers, you can find a root $3$ which works!
Then we have, \[(y-3)(\ldots)=y^3+y^2-2y-30.\]
Now, you can solve the problem  and the rest is a good exercise so left as an exercise.

Here we see another problem:
\begin{example}
Find all real solution to the eqaution,
\[x^4 +(x-2)^4=34.\]
\end{example}
It is a $4$ degree polynomial, so we have to think wisely.
An idea coomes that, we can rewrite the equation by folling, \[(y+1)^4+(y-1)^4=34\]
where, $x=y+1$ and so $x-2=y-1$ After doing 
some algebra we get a polynomial where all 
the degrees of $y$ is even, and assume $y^2=z$ 
then the polynomial is a quadratic and do some algebra.

\section{Polynomial Division}
You probably know that for a number $a$ we can find $q$ such that $q|a$ and we can write $a$ as follows:
\[a=bq+r\] where $0\le r<b$.

Similarly for a polynomial $a(x)$ we can find $b(x), q(x)$ and $r(x)$ such that,
\[a(x)=b(x)q(x) +r(x).\]
Here note that if $b(x)$ has degree $2$ then $r(x)$ is linear.

By using these facts we can solve a problem now:
\begin{example}
Find the remainder when the polynomial $x^{81}+x^{49}+x^{25}+x^9 +x$ is divided by the polynomial $x^3-x$.
\end{example}
 Here observe that we can write the polynomial by $Q(x)(x^3-x) +r(x)$ and also observe that $r(x)$ has degree $2$.
 So,
 \[x^{81}+x^{49}+x^{25}+x^9 +x=Q(x)(x^3-x) +ax^2+bx+c.\]
 Now we factor, \[x^3-x=x(x+1)(x-1).\]
 Now, pluging $x=0, x=1, x=-1$ in the above equation we get, \[a=0, b=5, c=0\]
 so, the remainder is $5x$.


 Always remember to write a polynomial $a(x)$ by $b(x)q(x)+r(x)$.

 Also remember the vieta's formulas. It  Assume that the polynomial 
 \[a_nx^n + a_{n-1}x^{n-1} + a_{n-2}x^{n-2}+ \ldots + a_2x^2 + a_1x + a_0 .\]
 has roots $r_1, r_2, r_3, \ldots , r_n$
  we have 
  \[ r_1+r_2+r_3+\ldots + r_n =-\frac{a_{n-1}}{a_n}\]
 and, \[r_1r_2 +r_2r_3 +\ldots +r_{n-1}r_n=\frac{a_{n-2}}{a_n}\]
 and similarly analogous...

 Here are going to solve a problem using the above proposition.

\begin{example}
Given that, $a+b+c>0$, $ab+bc+ca>0$ and $abc>0$. Prove that, $a,b,c>0$.
\end{example}
Here we will use method of contradiction!

%Assume that, 
We are going to construct a polynomial:
\[P(x)=x^3 -(a+b+c)x^2+(ab+bc+ca)x-abc\]
which has roots $a,b,c$.

And assume that $a$ is negative. But then $P(a)$ is negative which is supposed to be zero(as $a$ is a root of $P(x)$). Which is a contradiction!


\begin{example}
Let $n$ be a positive integer and for $1\le k \le n$ let $s_k$ be the sum of the products of the numbers $1, 1/2, 1/3,\ldots, 1/n$ taken $k$ at a time.
For example,
\[s_2 = 1\cdot \half + 1 \cdot \frac{1}{3} + \ldots \frac{1}{n-1}\cdot \frac{1}{n}\]
Find $s_1 +s_2 +\ldots + s_n$.
\end{example}
It is left as an exercise for the interested readers.
%\ldots

%left as an exercise
\begin{theorem}[Identity Theorem]
Two polynomials $f(x), g(x)$ have degree less than $n$. If they intersect at $n+1$ points ($x_1, x_2, \ldots x_{n+1}$) so, $f(x_1)=g(x_1), f(x_2)=g(x_2), \ldots $.

Then two polynomials are the same i.e. $f(x)=g(x)$.
\end{theorem}
Here let $P(x)=f(x)-g(x)$, it results that, $P(x)$ has $n+1$ roots which is not possible unless $P(x)$ is identically zero. (as $P(x)$ has degree at most $n$.)

Here, we solve a problem using the above result.
\begin{example}
If $f(x)$ is a monic quartic polynomial such that $f(-1)=-1, f(2)=-4, f(-3)=-9$ and $f(4)=-16$. Find $f(1)$.
\end{example}

$f(-1)=-1, f(2)=-4, f(-3)=-9$ and $f(4)=-16$
So, $f(x)=-x^2$ which is not a monic polynomial.

We asssume $f(x)=x^4 + bx^3 + cx^2 +dx +e$ as it is monic and quartic.

Then, consider the polynomial $f(x)+x^2=P(x)$ which is a 4 degree polynomial which has roots $-1, 2, -3, 4$.

At these points $P(x)$ is zero, and as it is a quartic so we have, \[P(x)=c(x+1)(x-2)(x+3)(x-4)\]
where $c$ is obviously zero as $f(x)$ is a monic.

\begin{example}
If $P(x)$ denotes a polynomial of degree $n$ such that \[P(k)=\frac{k}{k+1}\]
for $k=0,1,2,\ldots, n$. Determine $P(n+1).$
\end{example}
Here conisder a polynomial \[(x+1)P(x)-x=f(x)\]
by the above argument, we get, \[f(x)=cx(x-1)(x-2)(x-3)\cdots (x-n)\]
Plugging $x=-1$ we get, $1=c(-1)^{n+1}(n+1)!$
so, \[c=\frac{1}{(-1)^{n+1}(n+1)!}\]
The rest is a good exercise.

\section{Polynomial Interpolation}%, Polynomial Functional Equation}
Suppose you are given that $f(1)=7, f(2)=4, f(10)=-5$ and you are told to find the function. This process is called Polynomial Interpolation. The above problem asks you to find the polynomial which graph goes through the points $(1,7), (2,4), (10,-5)$.

\begin{example}
What is the next term of the sequence $2,5,11,23,\ldots $?
\end{example}
The answer is what you want!

We write $n$ term as $a_n$ and let \[a_n=(A_1 \times 2) + (A_2\times 5) + (A_3 \times 11) + (A_4 \times 23) + (A_5 \times 42)\]
Take $A_1, _2, A_3, A_4, A_5$ such that if we plug $n=1$ then we get $A_1=1$ and $A_2=A_3=A_4=A_5=0$. Similarly for $n=2,3,4,5$.
Magically take \[A_1=\frac{(n-2)(n-3)(n-4)(n-5)}{(1-2)(1-3)(1-4)(1-5)}\]

and analogous for $A_2, A_3 ,\ldots$

\begin{example}
$f$ be a function such that $f(1)=7, f(2)=4, f(10)=-5$
\end{example} 
From the above example it is easy. 
So,
\[f(x)=\frac{(x-2)(x-10)}{(1-2)(1-10)} \times 7 + \frac{(x-1)(x-10)}{(2-1)(2-10)} \times 4 + \frac{(x-1)(x-2)}{(10-1)(10-2)} \times (-5)\]
It is known as Lagrange Interpolation.
It is useful in finding an algebraic form of a sequence.

\begin{example}
\[f(x)=\frac{(x-b)(x-c)}{(a-b)(a-c)} + \frac{(x-a)(x-c)}{(b-a)(b-c)} + \frac{(x-a)(x-b)}{(c-a)(c-b)}\]
and $a\neq b \neq c$.
$\deg(f)\le 2$.
%What is the degree of $f$?
\end{example}

We get, $f(a)=f(b)=f(c)=1$. So, it is not possible to have degree 2. As it has same value at three points.

Consider $f(x)-1=p(x)$, and $p(a), p(b), p(c)$ is zero. So, $p(x)$ is identically zero and $f(x)=1$.

\begin{example}
Let $a,b,c,d$ be distinct real numbers. Show that \[\frac{a^4}{(a-b)(a-c)(a-d)}+\frac{b^4}{(b-a)(b-c)(b-d)}+\frac{c^4}{(c-a)(c-b)(c-d)}+\ldots=a+b+c+d.\]
%%%%%%%\[\frac{a^4}{(a-b)(a-c)(a-d)}+\frac{b^4}{(b-a)(b-c)(b-d)}+\frac{c^4}{(c-a)(c-b)(c-d)}+\frac{d^4}{(d-a)(d-b)(d-c)}=a+b+c+d.\]
\end{example}
\[P(x)=\frac{a^4(x-b)(x-c)(x-d)}{(a-b)(a-c)(a-d)}+\frac{b^4(x-a)(x-c)(x-d)}{(b-a)(b-c)(b-d)}+\frac{c^4(x-a)(x-b)(x-d)}{(c-a)(c-b)(c-d)}+\ldots\]

%%%%\[P(x)=\frac{a^4(x-b)(x-c)(x-d)}{(a-b)(a-c)(a-d)}+\frac{b^4(x-a)(x-c)(x-d)}{(b-a)(b-c)(b-d)}+\frac{c^4(x-a)(x-b)(x-d)}{(c-a)(c-b)(c-d)}+\frac{d^4(x-a)(x-b)(x-c)}{(d-a)(d-b)(d-c)}\]

The coefficient of $x^3$ in $P(x)$ 

\[f(x)=x^4-P(x)=(x-a)(x-b)(x-c)(x-d).\]

\section{Practice Problems}
\begin{problem}
Find all real x that satisfy the equation
\[ (x+1)(x^2+1)(x^3+1)= 30x^3 \]
\begin{hint}
\addhint{}
\addhint{}
\end{hint}
\end{problem}
\begin{problem}
Find all real solutions to $ x^4 +(x-2)^4 = 34$.
\begin{hint}
\addhint{}
\addhint{}
\end{hint}
\end{problem}
\begin{problem}
Find the reminder when $x^81 + x^49 +x^25 + x^9 +x $ is divided by the polynomial $x^3-x$
\begin{hint}
\addhint{}
\addhint{}
\end{hint}
\end{problem}
\begin{problem}
Let $P(x) = x^4 + ax^3 + bx^2 + cx + d$ where $a$, $b$, $c$ and $d$ are constants. If $P(1) = 10$, $P(2) = 20$ and $P(3)= 30$. Compute,
\[\frac{P(12)+ P(-8)}{10} \]
\begin{hint}
\addhint{}
\addhint{}
\end{hint}
\end{problem}
\begin{problem}
Find $x^2+y^2 $ if $x$ and $y$ are positive integers such that,
\[xy + x+ y = 71 \text{ and } x^2y +xy^2 =880\]
\begin{hint}
\addhint{}
\addhint{}
\end{hint}
\end{problem}
\begin{problem}
Let $n$ be a positive integer, for $1\le k \le n$, let $s_k$ be the sum of the products of the numbers $1, 1/2,1/3,  \ldots ,1/n$, taken $k $ at a time. For example, 
\[ s_2 = 1.\frac{1}{2} + 1.\frac{1}{3}+ \ldots +\frac{1}{n-1}.\frac{1}{n} \] 
Find $s_1 +\ldots s_n$.
\begin{hint}
\addhint{}
\addhint{}
\end{hint}
\end{problem}
\begin{problem}
Let $a$, $b$ and $c$ be real numbers such that $a+b+c>0$, $ab+bc+ca>0$ and $abc>0$. Prove that $a,b$ and $c$ are all positive
\begin{hint}
\addhint{}
\addhint{}
\end{hint}
\end{problem}
\begin{problem}
Let $r,s$ and $ t $ be the three roots of the equation
\[ 8x^3 + 1001x +2008=0\]
Find $(r+s)^3 + (s+t)^3 + (t+r)^3$.
\begin{hint}
\addhint{}
\addhint{}
\end{hint}
\end{problem}
\begin{problem}
If $f(x)$ is a monic polynomial such that $f(-1)=-1$, $f(2)=-4$, $f(-3)= -9$ and $f(4)=-16$. Find $f(1)$.
\begin{hint}
\addhint{}
\addhint{}
\end{hint}
\end{problem}
\begin{problem}
If $P(x)$ denotes a polynomia of degree $n$ such that \[P(k) = \frac{k}{k+1}\]
for $k= 0,1,\ldots ,n$, determine $P(n+1)$.
\begin{hint}
\addhint{}
\addhint{}
\end{hint}
\end{problem}
\begin{problem}
Let $a_1, a_2, a_3, \ldots, a_{10} $ be real numbers such that, 
\[\frac{a_1}{1}+ \frac{a_2}{2} +\ldots + \frac{a_{10}}{10} =1 \]
\[\frac{a_1}{2}+ \frac{a_2}{3} +\ldots + \frac{a_{10}}{11} =0 \]
\[\frac{a_1}{3}+ \frac{a_2}{4} +\ldots + \frac{a_{10}}{12} =0 \]
\[\ldots\]
\[\frac{a_1}{10}+ \frac{a_2}{11} +\ldots + \frac{a_{10}}{19} =0 \]
 Determine $a_1+a_2+ \ldots a_10$.

\begin{hint}
\addhint{}
\addhint{}
\end{hint}
\end{problem}
\begin{problem}
Let $a,b,c$ and $d$ be distinct real numbers. Show that 
\[\frac{a^4}{(a-b)(a-c)(a-d)}+\frac{b^4}{(b-a)(b-c)(b-d)}+\frac{c^4}{(c-a)(c-b)(c-d)}+\ldots=a+b+c+d.\]
\begin{hint}
\addhint{}
\addhint{}
\end{hint}
\end{problem}
\begin{problem}
A polynomial of degree $3n$ takes the value $0$ at $2,5,8,\ldots,3n-1$. The value $1$ at $1,4,7,\ldots, 3n-2$ and the value $2$ at $0,3, \ldots, 2n$ and its value at $3n+1$ is $730$. Find $n$.
\begin{hint}
\addhint{}
\addhint{}
\end{hint}
\end{problem}
\begin{problem}
Find all polynomials $P(x)$ with real coefficients that satisfy
\[(x-1)P(x)-xP(x-1)= x^2-x-1\]
for all real number $x$.
\begin{hint}
\addhint{}
\addhint{}
\end{hint}
\end{problem}
\begin{problem}
Suppose that $P(x)$ is a polynomial of odd degree satisfying 
\[P(x^2-1)=(P(x))^2 -1\]
for all $x$. Prove that $P(x)= x$ for all $x$.
\begin{hint}
\addhint{}
\addhint{}
\end{hint}
\end{problem}